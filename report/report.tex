%
% Refrence http://en.wikibooks.org/wiki/LaTeX
%

\documentclass[10pt]{article}

% ======
% Header
% ======

% use droid sans
\usepackage{droidsans}    
\renewcommand\familydefault{\sfdefault}   % set default font to sans-serif


% enable \href
\usepackage[colorlinks=true,
            linkcolor=blue,
            urlcolor=blue,
            allbordercolors={0 0 0},
            pdfborderstyle={/S/U/W 1}]{hyperref}

% enable \includegraphics for embedding jpgs
\usepackage{graphicx}  

% ==================
% Document Meta Data
% ==================

\title{Webtech Report}
\author{Imna Malik n James Sewart}
\date{29-5-16}

% ====
% Body
% ====

\begin{document}

    \maketitle

    \tableofcontents


    \begin{abstract}
        The \href{https://smple.uk}{Smple} (short for sample) web application is a visual geospatial music discovery service. Users can discover new artists playing in their area and sample their music using data gathered from the Internet.
    \end{abstract}

    \section{Client}
        \subsection{Style}
            The web app is designed to be used on modern browsers and is tested on the latest Safari for OS X, Firefox, Chrome, and Safari for iOS. We do not support old IE.
            \subsubsection{General Layout}
                \begin{figure}[!ht]
                  \centering
                    \reflectbox{
                      \includegraphics[height=60mm]{example1.png}}
                  \caption{Basic layout description.}
                \end{figure}

                The Smple web application consists of a single page filled with a map from the Google Maps API. There are two panels: the search on the left, and the information (info) on the right. 

                The search panel consists of an input to type a query and two date pickers to choose the date range for events to be displayed. When a search query is made results are shown underneath with their event date, band names, and the time to walk there.

                When an event is selected by clicking a marker or search result, the info panel slides into view from the right. At the top, the artist thumbnails are displayed. They are dynamically created depending on how many artists are playing in the selected event. Below, is the artist name, image, and biography for the selected artist. If the play button underneath is clicked, an embedded Spotify player is shown. Lastly is the events venue address and the link to the ticket-sale website which are contained in their individual panels.

                There are two markers present on the map: pink and purple. The pink markers represent events and the purple is used to indicate the current user location which is determined by the browser.

               If the map is moved or zoomed, the info panel will slide out of view. Additionally, when the info panel's content overflows the bottom of the page, it becomes scrollable.


            \subsubsection{Base Classes}
                The design of our layout is inspired by Material Design with it's iconic card styled boxes. Each panel is given a shadowbox class which gives the element a border radius, shadow, slight transparency, and a dark background colour. The search panel and the panels within the info panel are styled with this class.

            \subsubsection{Resizing}
                CSS media queries are used to change large scale layouts based on the screen width. There are three of these: mobile-size, large, and extra large. These changes involve setting the widths of the main panels such that they look good on the screen size. For the mobile screen this means the panels are full width and stack on top of each other. For the large size the info bar takes up a third of the screen, and the search 40\% of the screen.

                For a more dynamic resize control we subscribe to the onresize event of the window and scale the whole document's font size based on the screen width in JavaScript. Most elements are defined using ems and so they all scale accordingly.
                \begin{figure}[!ht]
                  \centering
                    \reflectbox{
                      \includegraphics[height=60mm]{example3.png}}
                  \caption{Example layout when rendered using a mobile browser.}
                \end{figure}

            \subsubsection{Info Panel}
                The info panel involves many different styling and scripting elements. Animations in the panel are done using the CSS transition property and setting the margin-right to 0 from -33\%. By default, moving this panel to the right would bring up a scrolling bar. However this is prevented by setting the overflow property to hidden.

                Initially, the animation's performance was poor in early development due interaction with the map styling. This caused a browser reflow to happen for every frame of the animation. Our research determined that we needed to avoid as many reflows as possible when animating for best performance. Changing the map to be position:absolute significantly improved performance as only repaints were necessary.

                In order to produce the desired scrolling effect, we set the overflow-y property to scroll. This allows the contents to scroll within its div. To ensure that the content does not increase the page size, the height is set to 100\% of the screen height.

                Artist thumbnails are contained within an element with text-align set to center. Each image is an inline-block to allow overflow and centering independent of the number of thumbnails. Since the images could be of any aspect ratio, we use the object-fit CSS property which is set as cover, and the left and top property which is set as 50\% to keep the thumbnail container filled and the image centered inside. The container is given a border radius of 50\% to create a circular thumbnail, and the overflow is set to hidden to crop the image inside. Furthermore, the thumbnail is animated to zoom on hover using a CSS selector that transforms the scale of the thumbnail container.

                Missing artists have a silhouette avatar picture as their thumbnail. This was created by finding a silhouette SVG found online and overlaying it on-top of a gradient PNG image which was imported into Inkscape with the image rendering mode set to smooth to ensure the best quality conversion. 

                The box underneath contains an element for the artist title, along with a paragraph tag for the biography. The title is displayed over the artist image and has a gradient behind it to make sure the text is visible. Additionally, the title is pinned to the bottom of the image. The image width is set to the parent's width so that only the height changes depending on the aspect ratio. This height pushes down the biography accordingly. For aesthetic purposes, there is a padding set on the biography.

                Beneath the artist information box is a purple SVG button. When clicked, it triggers a set of animations. The two triangles will turn 120 degrees clockwise with slightly different durations using SMIL animations. Also, the size of the box increases to reveal an iframe with a Spotify embedded player inside and the colour changes from purple to gray to match the player. These animations are reversed on a second click to hide the player. When updating the info panel, the URL for the iframe is also updated with the current artist's URL. This triggers the URL to be loaded inside it, causing the correct player to be displayed.

                Inkscape was used to create the SVG play button. First, the polygon tool was used to draw a white triangle path. Then by duplicating the triangle and enlarging it, we exclude the smaller triangle to produce stencil triangle. Then simply by overlaying the same-sized triangle and offsetting its position, we produce the final image.

                Lastly, there are two boxes which display venue and ticket information. The first displays the venue title and address. When this box is clicked, it will open a new tab with Songkick's venue page. The second box holds another custom ticket SVG. Similarly, if this box is clicked, the Songkick ticket purchasing site is opened.

                To produce the ticket SVG, we started with a rectangle with rounded corners. Afterwards, two rounded squares were created for each end and we then used the difference tool to cut in a semi-circle. We then duplicating a small rounded rectangle six times and combined them into one object using the combine tool for simplicity. Using this combined object, we exclude the shape from the ticket to produce a dotted line. This is the original (filled) ticket behind the stencil ticket. To produce the stencil, we first tried duplicating the SVG and using the outset tool to make one bigger. Then by aligning the two and using the exclusion tool, we produced a stencil ticket. This didn't quite have the desired shape so instead we scaled the second ticket and excluded these two. This produced an acceptable uniform path thickness. Then, by placing the stencil over a copy of the original and using the difference tool we cut a shadow. Now by simply placing the stencil in the gap, we produced the desired SVG. 

                The background image for the ticket box was found online and edited using GIMP. First, the image was cropped to be square-sized. Using the colour balance tool, we adjusted the shadows, midtones, and highlights to make the pink and blue stand out. Next, using the hue tool, we modify the pink to match the website's colour scheme. Other colours had their saturation and lightness reduced using this tool. Afterwards using the curve tool, we adjust the RGB channels so that the image is more aesthetically pleasing. To create the effect of the image fading to pink, we created a new pink layer with a layer mask. Using the gradient tool with the pink to black setting, we add a pink gradient to the mask. This mask is then applied to the layer.Afterwards, the same is done for the bottom to produce a black vignette effect. Furthermore, we then used the fuzzy tool with feathered edges to select the people in the image. By using the hue/saturation tool, the blues were adjusted to be more grey with the overlap set to a high value to consider overlapping hue values. Finally, we added the Songkick watermark as an overlying layer, and positioned the image in the bottom right corner.

                \begin{figure}[!ht]
                  \centering
                      \includegraphics[height=60mm]{example2.png}
                  \caption{Before and after gimpifying.}
                \end{figure}


                When a user hovers their mouse over the lower two boxes, an animation is triggered to dim the image. This is achieved using an anchor html tag that is given the class boxClickUrl that expands the anchor to cover the entire parent. The dim effect is achieved by setting the opacity of the image to a smaller value, revealing the black shadowbox behind.

            \subsubsection{Search Panel}
                From left to right in the search panel, we have a Google material icons search icon. Adjacent to the icon is an input box with placeholder text to indicate it is a search input. At the end, there are two date pickers produced using the \href{https://github.com/dbushell/Pikaday}{Pikaday} library. These allow the user to specify the query's date range.

                The previous items are positioned using the float CSS attribute so that each item is placed on the right of the previous one. Padding is set in various places to produce a pleasing result.  To create spaces between input boxes and text, non-breaking spaces are used. Date pickers, on hover, change the cursor to a pointer to indicate that they are clickable.

                For the search results, an empty list container is added for the search results to populate. A border-bottom is added to each list item as a border to separate results. Using the CSS selector last-child, we remove the last border-bottom so that the last list element does not have an end border.

                A custom calendar icon was created using two divs: the white-background class inside the red one. These are created using JavaScript, and it populates them with the required day and month.

                The date range picker uses a library for visualising calendars. Initially, this did not work in strict xhtml mode due to a bug in the library that we have fixed and filed a \href{https://github.com/dbushell/Pikaday/pull/526}{pull request} for. We initialised two of these calendars by specifying the default date range. Also, when a user clicks a date from the from calendar, all dates before the current date are disabled to prevent them from selecting events from the past.

            \subsubsection{Map}
                From the Google Maps API, the map has a custom theme set to match the colour scheme of the  website. Also, we use the Google map markers for our event visualiser but add our own custom SVG as their icons. To create these, a basic SVG marker was found. Next, by using Inkscape's \"break apart\" tool, we deleted the opaque center to produce a transparent hole in the center of the marker. Lastly, the The marker is animated using Google's \"BOUNCE" animation setting.

                When animating the markers using the Google Maps API we found that they would bounce in safari and Firefox but not Chrome. Stepping through the markers script led to a getElementsByTagName looking for tag HEAD in caps. Under XHTML tags are case sensitive and should always be lower case. In chrome and safari the browser understood that this call was looking for the lower-case head tag and could find it fine. In chrome this wasn't the case and the query would result in no results. Attempting to change the sites head tag to upper-case didn't solve the problem as other parts of googles script made queries to find tag head in lower case. The solution found for this problem was to have two heads, one upper-case and one lower. While this means that we fail to meet the XHTML spec, it means that the marker animations work and the page is still rendered fine in all the browsers tested.

        \subsection{Logic}
            There exists one script for all the logic of the webapp. We subscribe to the window onload event as well as the Google map callback. In the window load event we setup a few things, including setting up the callbacks for the SVG button, initialising the websocket to the server, and initialising the date filter boxes.

            When the map has loaded, we request the browsers location via the Google Maps API and setup the markers on the map. After both the map has loaded and the websocket has loaded we send the user location to the server via the websocket. This triggers the server to find nearby events to be send back to the client.

            When an array of events arrive we populate our internal events array and initialise the marker for each event. A query is made to find the walking time to each of the events as well. Once this has happened the user can click on the markers to open up the information for the event, as well as being able to search for events using the search bar.

            When a user clicks a marker, a class is swapped out on the sidebar so that it has a new margin-right property. As the infopanel has a transition property it will animate in. The same thing is done with the play button and Spotify player but with CSS properties background-color and padding-bottom to change the background colour of the play button and to show the Spotify player.

            When filling the info panel with event information we need to create a thumbnail for each artist. There can be any number of artists in an event so we create a div with class bandimage for each artist and append each one to the bandpics div, organising them properly. The Spotify player url is also updated to load the correct song.

            For the search bar, the keyup event is subscribed to for the input box so we can get a string representing the current search each time a new key is pressed. This event is used to update the search results. First the current query string is used to do a sublime style fuzzy matching on each of the first and second artist names in every event. The fuzziness refers to lack of need for sequential characters to match and instead there can be skips before matching. To visualise the matches the result of the search wraps each matched character with a bold html tag so that when the search results are displayed it can be clearly seen which of the artist names characters are matching.

            The results from the search include data such as the event date, first and second artist title, and walking time from the current location. For each result a list element is created that contains a visualisation of the date, ordering of artist names, and walking time.

            These list items are appended to a html list element that exists in the base html.

    \section{Server}
        The server uses the provided script with modifications. The index is served as a static file and dynamic content is achieved using Websocket connections between the client and server. This design choice simplifies the file resource serving logic, reducing the dynamic parts to message passing.

        Once the client has initialised the webpage, it sets up a secure Websocket connection to the server. A secure socket is preferred as non secure sockets are disallowed when viewing the page over https. The connection is abstracted into websocket.js such that server.js can simply pass a callback that is called when a new client query is made, with a parameter that can be called with the result to send back. server.js doesnt need to worry about managing connections.

        The client will query the browser for its current geographical location and when given, will send this to the server along with a desired date range. The server will reply at some point in the future with an array of events.

        The server will receive this data and use it to query the database using a geoNear query on a \href{https://docs.mongodb.com/manual/core/2dsphere/}{2dsphere} index, and a date range query on the event date. The results are sent to the client.

        As node doesn't come with a native websockets library a popular third party library was used. \href{https://github.com/theturtle32/WebSocket-Node}{theturtle32/WebSocket-Node}. Websockets provide a stream of messages, where a message is either some binary data or a string. We serialize JSON for sending data between the client and server. When the client sends its position and desired date range we serialize a data structure that looks like the following.

        \begin{verbatim}
        {
            pos: {
                lat: Number,
                lng: Number
            },
            dateRange: {
                from: Date,
                to: Date
            }
        }
        \end{verbatim}

        Dates aren't parsed into JavaScript date objects when using a JSON.parse so these are handled manually for the database queries.

        As well as giving the client the desired data we have, we also check whether the database needs updating. The data is collected using a variety of APIs. First we ask Songkick for the area that the user is in, this allows us to check the database to see if we have checked this area at all/recently. If we haven't got an up to date set of events for this area, we start downloading the events for the area as well as querying last.fm for the artists biography and picture, Spotify for a reference to their Spotify page, and Google's reverse geocode API to get the full address of the venue.

        For each area we limit the speed of requests in order to avoid rate limiting. Any returned errors by the APIs are handled graciously. Once complete data about an event is retrieved, it is stored in the database using an \href{https://docs.mongodb.com/manual/reference/method/db.collection.update/#upsert-option}{upsert} on the event title, to either update an already queried event, or insert as a new event.

        A possible optimization of the client-server interaction process would be to determine what events have already been sent to the client in order to avoid sending them again if they only scroll the map slightly. This would allow for significantly less data to be transferred between the two and would avoid reprocessing every event on the client side on every query change.

        As we don't store any non public data we don't require encryption for the data we store or obfuscation in any way. Validating the data we get from the client is necessary to stop a server crash. Location data is always sent over a secure websocket in order to prevent snooping.

    \section{Deployment}
        The server is written using some ES6 features and so requires a newer version of nodejs to run. By default the server connects to a local mongojs server with no special configuration added. This implementation favours root accessible vps' such as digital ocean or \href{https://aws.amazon.com/ec2/}{EC2}.

        The development workflow consisted of a digital ocean instance that was setup with a bare git repo. Git hooks were written so that when a push to the repo is made, a clone of the repo is updated and a call to npm install is made to update and install any libraries. On the server the server script is run using \href{https://github.com/remy/nodemon}{nodemon} so that when the repo is updated, the server will restart.

        This allows very fast staging tests, however this isn't suitable for a production run of the server. The site should be live at \href{https://smple.uk}{https://smple.uk}.

        Due to security issues some browsers wont connect a secure websocket with an invalid certificate making local development hard. The site is tested working on chrome locally for both https and http. When run under the correct domain everything works as it should.

    \section{API Usage}
        As we are using a few public APIs, we needed to take into consideration the terms for each of these.

        The Songkick API requires attribution when using their data in the form of a graphic they provide. We accommodated for this by adding their graphic into our ticket purchase button.

        Spotify have no requirement for the use of their embedded player as it already advertising where the player comes from.

        Last.fm require a link to their site on any information that is retrieved from them, we get a biography from them in which they include a link to their website so we are covered here.

\end{document}


